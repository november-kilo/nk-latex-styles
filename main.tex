\documentclass[openany]{memoir}
\usepackage{nk}
\usepackage{kantlipsum}

\title{\fontsize{56}{60}\usefont{OT1}{ppl}{m}{n}\selectfont NK \LaTeX{} styles}
\author{Natasha Korol\"{e}va}
\date{January 2019}

\begin{document}

\maketitle
\tableofcontents

\chapter{Introduction}
\epigraph{
\emph{Villian, I have done thy mother.} \\
Titus Andronicus --- Act IV, Scene ii}

\section{A section}

\kant[1]

\subsection{A subsection}

\kant[3]

\subsubsection{A subsubsection}

\kant[5]

\chapter{Vector calculus}

\section{Curl and divergence}\label{sect:curl-and-divergence}

The curl of the vector field $\nkvecfield{F}{x,y,z}$ is
\begin{equation}
\mathrm{curl} \nkvecfield{F}{x, y, z} =
\nkpar{\frac{\partial P}{\partial y} - \frac{\partial N}{\partial z}}\ihat
-
\nkpar{\frac{\partial P}{\partial x} - \frac{\partial M}{\partial z}}\jhat
+
\nkpar{\frac{\partial N}{\partial x} - \frac{\partial M}{\partial y}}\khat
\end{equation}
Using the cross product notation for curl
\begin{equation}
\nabla \bm{\times} \nkvecfield{F}{x, y, z} =
    \begin{bmatrix}
        \ihat & \jhat & \khat \\[0.6em]
        \frac{\partial}{\partial x} & \frac{\partial}{\partial y} & \frac{\partial}{\partial z} \\[0.6em]
        M & N & P\label{eq:curl}
     \end{bmatrix}
\end{equation}
When $\mathrm{curl} \vec{F} = 0$, the vector field is conservative.

\begin{mycode}
\begin{exmp}
Show that the vector field given by
$\nkvecfield{F}{x,y,z}=2xy\ihat+\nkpar{x^2+y^2}\jhat+2yz\khat$ is
conservative.
\nksolution
From \nkref{eq:curl}
\begin{align}
    \nabla \bm{\times} \nkvecfield{F}{x, y, z} &=
        \begin{bmatrix}
            \ihat & \jhat & \khat \\[0.6em]
            \frac{\partial}{\partial x} & \frac{\partial}{\partial y} & \frac{\partial}{\partial z} \\[0.6em]
            2xy & x^2+z^2 & 2yz \\[0.6em]
        \end{bmatrix} \\
    &=
        \begin{bmatrix}
            \frac{\partial}{\partial y} & \frac{\partial}{\partial z} \\[0.6em]
            x^2+z^2 & 2yz
        \end{bmatrix}\ihat
        -
        \begin{bmatrix}
            \frac{\partial}{\partial x} & \frac{\partial}{\partial z} \\[0.6em]
            2xy & 2yz
        \end{bmatrix}\jhat        
        +
        \begin{bmatrix}
            \frac{\partial}{\partial x} & \frac{\partial}{\partial y} \\[0.6em]
            2xy & x^2 + z^2
        \end{bmatrix}\khat \\[0.6em]
    &= \nkpar{2z-2z}\ihat - \nkpar{0-0}\jhat + \nkpar{2x-2x}\khat \\
    &= 0
\end{align}
\end{exmp}
\end{mycode}

The divergence of the vector field $\nkvecfield{F}{x, y, z}$ is
\begin{equation}
    \mathrm{div} \nkvecfield{F}{x,y,z}=
    \frac{\partial M}{\partial x} + \frac{\partial N}{\partial y} + \frac{\partial P}{\partial z}\label{eq:divergence}
\end{equation}
Using the dot product notation for divergence
\begin{equation}
    \mathrm{div}\vec{F} = \nabla \bm{\cdot} \vec{F}
\end{equation}

\begin{mycode}
\begin{exmp}
Given the vector field $\nkvecfield{F}{x,y,z}=\sin x\ihat + \cos y\jhat + z^2\khat$,
find its divergence.
\nksolution
From \nkref{eq:divergence}
\begin{align}
    \mathrm{div} \nkvecfield{F}{x,y,z}&=\frac{\partial M}{\partial x} + \frac{\partial N}{\partial y} + \frac{\partial P}{\partial z} \\
    &= \cos x - \sin y + 2z
\end{align}
\end{exmp}
\end{mycode}

\begin{nkremarkgood}
The curl of a vector field is a vector field.
If $\mathrm{curl}\vec{F}=0$, then the vector field is called irrotational.
\end{nkremarkgood}

\begin{nkremarkgood}
The divergence of a vector field is a scalar function.
If $\mathrm{div}\vec{F}=0$, then the vector field is called imcompressible.
If the divergence is positive, then it is called a source.
If the divergence is negative, then it is called a sink.
\end{nkremarkgood}

\chapter{Code syntax highlighting}

\section{LPC}

An example of DGD LPC.

\shaded
\begin{Verbatim}[commandchars=\\\{\}]
    \nkcomment{// Vector.c}

    \nkkeyword{private\ mixed}\ i\nkpunctuation{;}
    \nkkeyword{private\ mixed}\ j\nkpunctuation{;}
    \nkkeyword{private\ mixed}\ k\nkpunctuation{;}

    \nkkeyword{private}\ \nkkeyword{int}\ \nkfunction{validComponents}\nkpunctuation{(}\nkkeyword{mixed}\ \nkoperator{*}components\nkpunctuation{)}\ \nkpunctuation{\{}
        \nkkeyword{return}\ \nkconstant{TRUE}\nkpunctuation{;}
    \nkpunctuation{\}}

    \nkkeyword{Vector}\ \nkfunction{differentiate}\nkpunctuation{(}\nkpunctuation{)}\ \nkpunctuation{\{}
        \nkkeyword{return}\ new\ \nkfunction{Vector}\nkpunctuation{(}i\nkoperator{->}\nkfunction{differentiate}\nkpunctuation{(}\nkpunctuation{)}\nkpunctuation{,}\ j\nkoperator{->}\nkfunction{differentiate}\nkpunctuation{(}\nkpunctuation{)}\nkpunctuation{,}\ k\nkoperator{->}\nkfunction{differentiate}\nkpunctuation{(}\nkpunctuation{)}\nkpunctuation{)}\nkpunctuation{;}
    \nkpunctuation{\}}

    \nkkeyword{static}\ \nkkeyword{void}\ \nkfunction{create}\nkpunctuation{(}\nkkeyword{mixed}\ \nkoperator{*}components\nkpunctuation{)}\ \nkpunctuation{\{}
        \nkkeyword{if}\ \nkpunctuation{(}\nkfunction{validComponents}\nkpunctuation{(}components\nkpunctuation{)}\nkpunctuation{)}\ \nkpunctuation{\{}
            \nkpunctuation{(}\nkpunctuation{\{}\ i\nkpunctuation{,}\ j\nkpunctuation{,}\ k\ \nkpunctuation{\}}\nkpunctuation{)}\ \nkoperator{=}\ components\nkpunctuation{;}
        \nkpunctuation{\}}
    \nkpunctuation{\}}
\end{Verbatim}
\endshaded

\end{document}
